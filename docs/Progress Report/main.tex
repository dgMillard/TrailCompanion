\documentclass[letterpaper, 10pt,titlepage]{article}

\usepackage[utf8]{inputenc}
\usepackage [english]{babel}
\usepackage{graphicx}                                        
\usepackage{amssymb}                                         
\usepackage{amsmath}                                         
\usepackage{amsthm}                                          
\usepackage{alltt}                                           
\usepackage{float}
\usepackage{url}
\newcommand\tab[1][1cm]{\hspace*{#1}}
\setlength{\parindent}{0em}
\setlength{\parskip}{1em}
\usepackage[letterpaper, margin=0.75in]{geometry}
\usepackage{balance}
\usepackage[TABBOTCAP, tight]{subfigure}
\usepackage{enumitem}
\usepackage{hyperref}
\hypersetup{
  colorlinks = true,
  linkcolor  = black
}
\usepackage{listings}
\usepackage{color}
 
\definecolor{codegreen}{rgb}{0,0.6,0}
\definecolor{codegray}{rgb}{0.5,0.5,0.5}
\definecolor{codepurple}{rgb}{0.58,0,0.82}
\definecolor{backcolour}{rgb}{0.95,0.95,0.92}
 
\lstdefinestyle{mystyle}{
    backgroundcolor=\color{backcolour},   
    commentstyle=\color{codegreen},
    keywordstyle=\color{magenta},
    numberstyle=\tiny\color{codegray},
    stringstyle=\color{codepurple},
    basicstyle=\footnotesize,
    breakatwhitespace=false,         
    breaklines=true,                 
    captionpos=b,                    
    keepspaces=true,                 
    numbers=left,                    
    numbersep=5pt,                  
    showspaces=false,                
    showstringspaces=false,
    showtabs=false,                  
    tabsize=2
}
 
\lstset{style=mystyle}




\setcounter{secnumdepth}{4}
\def\name{Jiawei Liu}

\hypersetup{
  colorlinks = true,
  urlcolor = black,
  pdfauthor = {\name},
  pdfkeywords = {Problem Statement},
  pdftitle = {Capstone Project},
  pdfsubject = {Capstone Project},
  pdfpagemode = UseNone
}



\begin{document}

\begin{titlepage}
\begin{center}
    \Huge
    \textbf{Progress Report}\\
    \textbf{Capstone Project}\\
    \vspace{1.0cm}
    \large
    Developers: Charles Henninger, Duncan Millard, Jiawei Liu\\
    Sponsor: Nancy Hildebrandt\\
    \vspace{1.5cm}
    \large
    Instructor: D. Kevin McGrath\\

    \large
    CS 461, Fall 2016, Oregon State University\\    

    \vspace{3.2cm}

    \large
    \underline{Abstract}\\
    \vspace{0.3cm}
    \end{center}
    \large

    \tab The Santiam wagon trail is historic trail located in the Willamette National Forest. The local ranger stations wish to have a mobile app that is capable of taking users on a tour of the wagon trail without needing a connection to the internet. The app we will be self guided, and provide waypoints along the trail that contain information on the area, videos, and other points of interest. To accomplish this without an internet connection, we will be using packages featuring different aspects of the trail, including videos and text documents, that can be downloaded prior to arrival at the trail head. This document details our progress in developing this project over the last ten weeks, including the small problems that we’ve encountered and our plan for the next few weeks leading up to the start of our second senior design class.
    
    \vspace{0.8cm}
    \vfill
    
\begin{center}    
    Dec 6, 2016

\end{center}
\end{titlepage}





\section{Progress report}
The Santiam wagon trail is historic trail located in the Willamette National Forest. The local ranger stations wish to have a mobile app that is capable of taking users on a tour of the wagon trail without needing a connection to the internet. The app we will be self guided, and provide waypoints along the trail that contain information on the area, videos, and other points of interest. To accomplish this without an internet connection, we will be using packages featuring different aspects of the trail, including videos and text documents, that can be downloaded prior to arrival at the trail head. We are now at the end of the tenth week of our nine month development period. We have completed multiple informational tech and design documents, and gotten a solid plan for what our project is going to look like after development is completed. These plans have been shown and signed off on by our client, and we will be moving on to the actual development of the different project platforms within the next few weeks. Once we come back for the second of three terms of our senior design class, we will have some code completed, and will continue with creating documents and further plans for our project. 
The last few weeks of development have gone extremely smooth, and we have experienced little to no issues beyond minor scheduling conflicts. Our client is responsive and supportive, and has enough technical experience to make communication effective and easy. The most difficult part of this development phase by far was deciding on what tools we were going to use, mainly because we had to find tools that had available licensing that made them viable to students with little monetary resources to work with. This team is optimistic of our future progress considering the fantastic progress made in the last few months. 

It’s fair to say that our development cycle began week three of this term, during our second sit down meeting with our client, Nancy Hildebrandt. During this sitdown, Nancy informed us on the desired goals and side goals of this project, as well as providing some useful information that helped us form a better idea of what the final product of this project would look like.
Week four was our first meeting with our teaching assistant, Xinze, who gave us some useful information on our first project document, the Client Requirement document. The rest of this week was spent forming an outline of this document using the information provided by our client the previous week.  

The following week, week six, we worked almost exclusively on the Client Requirement document, as well as researching what tools we would be using for our project once development started in earnest.

After the Client Requirement document was finished at the end of week six, we sent it over to Nancy during week seven to be approved and signed off on. The rest of week seven was spent preparing for our Tech review document using the research on what tools we would potentially use. 

Week eight began with a trip out to the trail sight with our client and her associates to familiarize ourselves with the area that we would be working with. Later in the week we completed the Tech review document and have begun to plan for the completion of our Design document due during week 10.

Week nine was spent doing research individually due to the holiday break. 

Week ten was spent entirely on completing our Design document, using both the information from our tech review document and previous information about our project gathered from our client. We submitted the completed design document to Nancy for approval, and confirmed a more concrete outline for what our project solution will be. 





\section{Table}
\begin{tabular}{|p{0.3\linewidth}|p{0.3\linewidth}|p{0.3\linewidth}|}
\hline
\centering Positives &
\centering Deltas &   
\centering Actions \tabularnewline
\hline
-Team interactions are extremely easy to deal with and teamwork comes naturally to the group. 
\vspace{0.2cm}

-Team skill sets are varied and fit perfectly into the required platforms for this project& 

-Scheduling needs to be handled further in advance to prevent last minute completions of documents.& 

-One round of scheduling should be done early in the development term to create an awareness of what we need to complete and when. 
\tabularnewline
\hline
\end{tabular}










































\end{document}
